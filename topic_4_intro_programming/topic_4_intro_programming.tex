\chapter{Computational Thinking, Problem Solving, and Programming}

\section{Introduction to Programming}

\subsection*{Fundamental Operations}

\subsection*{Operations}

\subsection*{Programming Languages}

In order for any computer to work, it must be programmed. Programming a computer simply means inputting instructions that the machine is to follow so that it can solve whatever problem the programmer wants it to solve. In the early days (1940s), programming a computer meant manually wiring the machine with patch cables and wires so that data and output was channeled to where it was supposed to go. As you can imagine, such a process was time-consuming, laborious, error-prone, and expensive. In fact, once the computer was programmed to solve the problem, it could only solve that one problem. The machine had to be rewired in order to solve a new problem. Yuck!

In later models, computers could be programed using a series of punched cards. A punch card contained holes that represented 1s and 0s in binary code. These holes represented either the data to be manipulated, or the wires and circuits within the computer that needed to be activated in order to process the data. The output was also written to punch cards.  This process was still laborious, but much easier than wiring the computer over again. In addition, to solve a new problem, all the user had to do was load a new set of punch cards. The more complicated the program, the more punch cards one needed. 

As computers because smaller, cheaper, and more powerful thanks to the invention of the transistor and microchip, the expectations people made on them increased as well. Computers began to be programmed not with punch cards but through a language in which a person inputed instructions using numbers and letters and these instructions were then translated into instructions the computer can process. This process is called compiling and the program that does it is called a compiler.

These rules that govern how these instructions are to be written and inputted into the computer make what is called a programming language. Programming languages come in two broad flavors - low level and high level languages. Low level languages use numbers and letters to represent instructions and data for the computer. However, these instructions are written directly for the CPU and thus can be very hard for someone not familiar with it to understand. Assembly is an example of a low level language. Here is an example of assembly code:


%Example assmebly code

As users made more demands on computers as their power increased, so did the complexity of the programs required. Using a low-level language like Assembly to accomplish these this was getting too complicated and error prone. Thus came the need for high-level languages. A high-level computer language also has grammar and syntax, but is much more human readable. (That does not mean easy, it just means that the instructions are much easier to understand.)



%Mention Grace Hopper and compiler.
%Some videos on how computers were loaded would be a good idea.
%Adding some images of punch cards and wiring would also be helpful.

\section{Using Programming Languages}

Define variable, constant, operator, object
Define operators
Analyze use of variables, constants, and operators in algorithms
Construct algorithms using loops and branching.

