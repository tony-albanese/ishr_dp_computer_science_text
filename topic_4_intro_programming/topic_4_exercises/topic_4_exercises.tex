%These should be in an exercise environment. They should appear at the end of the chapter and the solutions at the end.
\section{Past Paper Problems}
\begin{exercise*} %N16HLP1-Q6
Consider the following recursive algorithm FUN(X, N), where X and N are two integers.	
	\begin{verbatim}
	FUN(X, N)
	if N<=0 then
	return 1
	else
	return X*FUN(X, N-1)
	end if
	\end{verbatim}
	The return statement gives the value that the algorithm generates.
	
	\begin{parts}
		\item Determine how many times multiplication is performed when this algorithm is executed.
		\begin{solution}
			N;
		\end{solution}
		\item Determine the value of FUN(2,3), showing all of your working.
		\begin{solution}
			Here's the solution.
			\begin{verbatim}
			FUN(2,3)=
			=2 * FUN(2,2) ;
			=2 * 2*FUN(2,1) ;
			=2 * 2 * 2 * FUN(2,0);
			=2*2*2*1= 8;
			\end{verbatim}
		\end{solution}
		\item State the purpose of this recursive algorithm.
		\begin{solution}
			Solution
		\end{solution}
	\end{parts}
\end{exercise*}



\begin{exercise*}
A new higher level programming language is being developed.
\begin{parts}
\item Identify two reasons why consistent grammar and syntax should be essential features of a higher level programming language.
\item Identify two features of a user interface that will allow application programmers to interact more easily with the programming language.
\item State one method of providing user documentation.
\end{parts}
\end{exercise*}


\begin{exercise*}
Application programmers who use this programming language will be able to choose to use either an interpreter or a compiler.
\begin{parts}
\item Outline the need for an interpreter or a compiler.
\item Describe one advantage to application programmers of having both an interpreter and a compiler available.
\end{parts}
One of the predefined sub-programs in the new language is sumOdd(). It accepts an integer N as input. If $N <= 0$ it outputs -1, otherwise it outputs the sum of the first N odd numbers.
For example:
sumOdd(4) outputs 16, because 4 is not less than 0, and 1 + 3 + 5 + 7 = 16.
sumOdd(-3) outputs -1, because -3 is less than 0.
\begin{parts}
\item Construct, in pseudocode, the algorithm for sumOdd().
\item Outline the need for predefined sub-programs and collections.
\end{parts}
\end{exercise*}

\begin{exercise*}%N16HLP1-Q12
A two-dimensional array, A, has N rows and N columns, where N is a positive integer.
The following algorithm is written to fill array A with the numbers $1, 2, 3,\ldots N^2$.
\begin{verbatim}
N=input("Enter an integer greater than zero")
K=N*N
	loop for ROW=0 to N-1
		loop for COLUMN=0 to N-1
		A[ROW][COLUMN]=K
		K=K-1
		end loop
	end loop
\end{verbatim}
\begin{parts}
\item Trace the algorithm, with an input of $N=3$, to show the contents of array A after the algorithm has been executed.
\end{parts}
There are many different ways of placing the numbers 1 to $N^2$ into an $N \times N$ two-dimensional array. 
The following two-dimensional array, with dimensions $5 \times 5$ has been filled in a circular (spiral) pattern with numbers 1 to $5^2$.

\includegraphics[scale=0.6]{topic_4_intro_programming/topic_4_exercises/array}
\newline

The general process of filling an $N \times N$ two-dimensional array, in a circular (spiral) pattern, with numbers from 1 to $N^2$ could be described as follows:

\begin{itemize}
\item initialize Z=1,
\item initialize TOP, BOTTOM, LEFT and RIGHT,
\item iterate until the whole array is filled,
\item each time Z is placed correctly increase the value of Z by 1,
\item fill the elements of the TOP row starting from LEFT to RIGHT,
\item increase TOP by 1 before filling the elements of the RIGHT column,
\item fill the elements of the RIGHT column starting from TOP to BOTTOM,
\item decrease RIGHT by 1 before filling the elements of the BOTTOM row,
\item and continue filling the BOTTOM row and LEFT column in a similar way, adjusting TOP, RIGHT, BOTTOM and LEFT accordingly.
\end{itemize}

\begin{parts}
\item State the initial values for TOP, BOTTOM, LEFT and RIGHT.
\item State the consequence of not increasing TOP by 1 before starting to fill the elements of the RIGHT column.
\item In the algorithm described above, state the indices (subscripts) of the first and the last element to be filled in the BOTTOM row.
\item Construct, in pseudocode, an algorithm to fill an $N \times N$ two-dimensional array, in a circular (spiral) pattern, with numbers from 1 to $N^2$ as described above.
\end{parts}
\end{exercise*}