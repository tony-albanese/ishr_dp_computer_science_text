%These should be in an exercise environment.

Consider the following recursive algorithm FUN(X, N), where X and N are two integers.
FUN(X, N)
if N<=0 then
return 1
else
return X*FUN(X, N-1)
end if
The return statement gives the value that the algorithm generates.
(a) Determine how many times multiplication is performed when this algorithm is executed. [1]
(b) Determine the value of FUN(2,3), showing all of your working. [3]
(c) State the purpose of this recursive algorithm.





A new higher level programming language is being developed.

(a) Identify two reasons why consistent grammar and syntax should be essential features

of a higher level programming language. [2]

(b) Identify two features of a user interface that will allow application programmers to

interact more easily with the programming language. [2]

(c) State one method of providing user documentation. [1]

Application programmers who use this programming language will be able to choose to use

either an interpreter or a compiler.

(d) (i) Outline the need for an interpreter or a compiler. [2]

(ii) Describe one advantage to application programmers of having both an

interpreter and a compiler available. [2]

One of the predefined sub-programs in the new language is sumOdd(). It accepts an integer

N as input. If N<=0 it outputs -1, otherwise it outputs the sum of the first N odd numbers.

For example:

sumOdd(4) outputs 16, because 4 is not less than 0, and 1 + 3 + 5 + 7 = 16.

sumOdd(−3) outputs −1, because −3 is less than 0.

(e) Construct, in pseudocode, the algorithm for sumOdd(). [4]

(f)	 Outline the need for predefined sub-programs and collections.






A two-dimensional array, A, has N rows and N columns, where N is a positive integer.

The following algorithm is written to fill array A with the numbers 1, 2, 3,…, N2.

N=input(‘Enter an integer greater than zero’)

K=N*N

loop for ROW=0 to N-1

loop for COLUMN=0 to N-1

A[ROW][COLUMN]=K

K=K-1

end loop

end loop


Trace the algorithm, with an input of N=3, to show the contents of array A after the

algorithm has been executed.


There are many different ways of placing the numbers 1 to N2 into an N × N two-dimensional array.

The following two-dimensional array, with dimensions 5 × 5 has been filled in a circular

(spiral) pattern with numbers 1 to 52.




The general process of filling an N × N two-dimensional array, in a circular (spiral) pattern,

with numbers from 1 to N2 could be described as follows:

• initialize Z=1,

• initialize TOP, BOTTOM, LEFT and RIGHT,

• iterate until the whole array is filled,

• each time Z is placed correctly increase the value of Z by 1,

• fill the elements of the TOP row starting from LEFT to RIGHT,

• increase TOP by 1 before filling the elements of the RIGHT column,

• fill the elements of the RIGHT column starting from TOP to BOTTOM,

• decrease RIGHT by 1 before filling the elements of the BOTTOM row,

• and continue filling the BOTTOM row and LEFT column in a similar way,

adjusting TOP, RIGHT, BOTTOM and LEFT accordingly.

(b) (i) State the initial values for TOP, BOTTOM, LEFT and RIGHT. [1]

(ii) State the consequence of not increasing TOP by 1 before starting to fill the

elements of the RIGHT column. [1]

(iii)	 In the algorithm described above, state the indices (subscripts) of the first and

the last element to be filled in the BOTTOM row. [1]

(c)	 Construct, in pseudocode, an algorithm to fill an N × N two-dimensional array, in a

circular (spiral) pattern, with numbers from 1 to N2 as described above.