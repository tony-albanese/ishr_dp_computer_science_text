\chapter{Dummy Chapter}
The chapter should start with perhaps a quote followed by the main objective of the unit. 

The \textbf{objective} of the unit should be specially formatted so that is visually stands out from the rest of the text.


\section{Subtopics}
This is where the meat of the chapters lives. Each section should be a standalone section of text that makes sense. 
	\begin{positiveInformation}{Positive Information}
		This is an environment for screening positive information. In other words, this is what the user ought to do and is for best practices.
	\end{positiveInformation}
There should also be exercises based on past papers which students can practice their work.
%This is an exercise environment. All of the exercises should go here.
\newline
Dummy exercise package.

\begin{exercise}
Tell me what the point of computer science is.
	\begin{solution}[]
	The point of computer science is to learn about computers!
	\end{solution}
\end{exercise}


\begin{exercise}
	
	What is going on here?
	
	\begin{solution}[]
	
	I have no idea.
	
	\end{solution}
\end{exercise}


\section{More subtopics}

The subsections continue. And they will continue. Made some more changes.
It is also important to have an environment where students can have negative information.

\begin{negativeInformation}{Worst practices}
	It is also important to avoid the following mistakes in your code base.
	\begin{itemize}
		\item Null pointers
		\item Exposing your variables
		\item Not commenting your code.
	\end{itemize}
	
\end{negativeInformation}


\section{Past Paper Questions}

Each section should end with past paper questions. These could be built into the chapter or this could be a link to online resources.

%This document contains problem sets for the end of the chapter.
\begin{problem}
	This is a question A in a problem environment.
	\begin{solution}
		This is the solution to problem A in the problem environment.
	\end{solution}
\end{problem}

\begin{problem}
	This is question B in a problem environment.
	\begin{solution}
		This is the solution to problem B in a problem environment.
	\end{solution}
\end{problem}